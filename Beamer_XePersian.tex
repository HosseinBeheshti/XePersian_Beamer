\documentclass{beamer}

\usetheme{Warsaw}

%\usetheme{Singapore}
%%% اینها هنوز کار نمی‌کنند.
%%\usetheme{Bergen}
%%\usetheme{AnnArbor}
%%\usetheme{CambridgeUS}
%%\usetheme{Hannover}
%%%\usetheme{Boadilla}
%%%\usetheme{Madrid}
%%%\usetheme{Antibes}
%%%\usetheme{Copenhagen}\setbeamercovered{dynamic}
%%% \usetheme{Marburg} 
%%%%%%%%%%%%%%%%%%%%%%%%%%%%%

%در ورژن جدید زی‌پرشین برای تایپ متن‌های ریاضی، این سه بسته، حتماً باید فراخوانی شود
\usepackage{amsthm,amssymb}
\usepackage{float}
\usepackage{graphicx}
\usepackage{standalone}
\usepackage{tikz}
\usetikzlibrary{shapes,arrows}
\usepackage[ruled]{algorithm}
\usepackage{algorithmic}
\usepackage{pgfplots}
\pgfplotsset{compat=newest}
\usepackage{standalone}
\usepackage{bm}
%%%%%%%%%%%%%%%%%%%%%%%%%%%%%
\usefonttheme{serif}

\usepackage{ptext}
\usepackage{xepersian}
\settextfont{Yas}

\include{tashih}
\include{commands}


\addtobeamertemplate{navigation symbols}{}{%
    \usebeamerfont{footline}%
    \usebeamercolor[fg]{footline}%
    \hspace{1em}%
    \insertframenumber/\inserttotalframenumber
}
\setbeamercolor{footline}{fg=blue}
\setbeamerfont{footline}{series=\bfseries}

\begin{document}

\title{بازیابی وفقی سیگنال‌های دیکشنری تنک با استفاده از نمونه‌های باینری}
\subtitle{}
\author{حسین بهشتی}


\begin{frame}
\maketitle
\end{frame}

%%% زدن \hfill  در دستور \section لازم است.
%%%%%%%%%%%%%%%
\begin{frame}
\tableofcontents
\end{frame}

%%%%%%%%%%%%%%%%%%%%%%%%
%%%%%%%%%%%%%%%%%%%%%%%%
%%%%%%%%%%%%%%%%%%%%%%%%
%%%%%%%%%%%%%%%%%%%%%%%%

\section{حسگری فشرده\hfill}

\begin{frame}
\frametitle{حسگری فشرده}

\end{frame}

\begin{frame}
\frametitle{حسگری فشرده}
\framesubtitle{حسگری فشرده کوانتیزه}
\end{frame}

\begin{frame}
\frametitle{حسگری فشرده}
\framesubtitle{حسگری فشرده تک بیتی}
\end{frame}

\begin{frame}
\frametitle{حسگری فشرده}
\framesubtitle{حسگری فشرده تک بیتی}
\end{frame}

%%%%%%%%%%%%%%%%%%%%%%%%
%%%%%%%%%%%%%%%%%%%%%%%%
%%%%%%%%%%%%%%%%%%%%%%%%
%%%%%%%%%%%%%%%%%%%%%%%%

\section{سیگنال‌های دیکشنری تنک\hfill}

\begin{frame}
\frametitle{سیگنال‌های دیکشنری تنک}

\end{frame}

\begin{frame}
\frametitle{سیگنال‌های دیکشنری تنک}

\end{frame}
%%%%%%%%%%%%%%%%%%%%%%%%
%%%%%%%%%%%%%%%%%%%%%%%%
%%%%%%%%%%%%%%%%%%%%%%%%
%%%%%%%%%%%%%%%%%%%%%%%%

\section{هندسه‌ی ابعاد بالا\hfill}

\begin{frame}
\frametitle{هندسه‌ی ابعاد بالا}

\end{frame}

\begin{frame}
\frametitle{هندسه‌ی ابعاد بالا}
\framesubtitle{\lr{Random Hyperplane Tessellations}}
\end{frame}

\begin{frame}
\frametitle{هندسه‌ی ابعاد بالا}
\framesubtitle{\lr{Random Hyperplane Tessellations}}
\end{frame}

%%%%%%%%%%%%%%%%%%%%%%%%
%%%%%%%%%%%%%%%%%%%%%%%%
%%%%%%%%%%%%%%%%%%%%%%%%
%%%%%%%%%%%%%%%%%%%%%%%%

\section{الگوریتم پیشنهادی\hfill}

\begin{frame}
\frametitle{مدل سیستم}

\begin{figure}[t]
	\centering
	\includestandalone[scale=0.6]{Images/AdaptiveBD}
	%% Tikz File 'mytikz.tex'
\documentclass{standalone}

%\usetikzlibrary{...}
\begin{document}
\tikzstyle{block} = [draw, fill=blue!20, rectangle, 
minimum height=3em, minimum width=6em]
\tikzstyle{sum} = [draw, fill=blue!20, circle, node distance=2cm]
\tikzstyle{input} = [coordinate]
\tikzstyle{output} = [coordinate]
\tikzstyle{pinstyle} = [pin edge={to-,thin,black}]
\begin{tikzpicture}[auto, node distance=2cm,>=latex']
% We start by placing the blocks
\node [input, name=input] {};

\node [block, right of=input,node distance=3cm] (CoeS) {
	$\begin{array}{c}
	\bm{x} \\
	\bm{D}^{\ast}\bm{f} 
	\end{array}	$};

\node [block, right of=CoeS,node distance=4cm] (SigS) {$ \bm{f} $};

\node [block, right of=SigS,node distance=3cm] (system) {$ \bm{A}^{(i)}\bm{f} $};

\node [sum, right of=system] (sum) {};
    
\node [block, right of=sum,node distance=3cm] (Sign) { $ \text{sign}\left(\bm{A}^{(i)}\bm{f}- \bm{\varphi}^{(i)}\right) $};

\node [block, below of=Sign,node distance=1.5cm] (ATh) { $ \bm{\varphi}^{(i)} $};

\node [output, right of=Sign] (yTemp) {};
\node [output, right of=yTemp] (yOut) {};

% We draw an edge between the controller and system block to 
% calculate the coordinate u. We need it to place the measurement block. 
\draw [->] (CoeS) -- node {$\rightarrow\bm{D}$} (SigS);

\draw [->] (SigS) -- node {$\bm{D}^{\ast}\leftarrow$} (CoeS);

\draw [->] (SigS) -- node {$\bm{A}$} (system);

\draw [draw,->] (system) -- node {$+$} (sum);

\draw [draw,->] (sum) -- node {} (Sign);

\draw [draw,-] (Sign) -- node {} (yTemp);

\draw [draw,->] (yTemp) -- node {$ \bm{y} $} (yOut);


\draw [->] (yTemp) |- (ATh);

\draw [->] (ATh) -| node[pos=0.99] {$-$} node [near end] {} (sum);

%\draw [->] (controller) -- node[name=u] {$u$} (system);
\end{tikzpicture}

\end{document}
	\caption{میی}
	\label{fig:AdaptiveBD}
\end{figure}
\end{frame}

%%%%%%%%%%%%%%%%%%%%%%%%
%%%%%%%%%%%%%%%%%%%%%%%%
%%%%%%%%%%%%%%%%%%%%%%%%
%%%%%%%%%%%%%%%%%%%%%%%%

\section{نتایج شبیه‌سازی\hfill}

\begin{frame}
\frametitle{نتایج شبیه‌سازی}
\framesubtitle{مقایسه‌ی الگوریتم‌ها}
\end{frame}

%%%%%%%%%%%%%%%%%%%%%%%%
%%%%%%%%%%%%%%%%%%%%%%%%
%%%%%%%%%%%%%%%%%%%%%%%%
%%%%%%%%%%%%%%%%%%%%%%%%

%%%%%%%%%%%%%%%

\end{document}