\section{قسمتی که در اینجاست\hfill}
\subsection{زیرقسمتی که در اینجاست}
\subsubsection{زیرزیرقسمتی که در اینجاست}
\begin{frame}
\tableofcontents
\end{frame}

\begin{frame}
\frametitle{عنوان اسلاید}
\framesubtitle{زیر عنوان اسلاید}
\end{frame}

\begin{frame}
\begin{enumerate}
\item 
این یک متن است که در اینجا قرار می‌دهیم.
\end{enumerate}

\begin{itemize}
\item 
این یک متن است که در اینجا قرار می‌دهیم.
\end{itemize}
\end{frame}


\begin{frame}
متن%
\footnote{این یک زیرنویس پارسی است.}%

متن%
 \LTRfootnote{This is a latin footnote.}%
 
متن%
\RTLfootnote{این هم زیرنویس پارسی دیگری است.}

\end{frame}

\begin{frame}
\ptext[1]
\begin{example}
این یک مثال است.
\end{example}

\begin{definition}
این یک تعریف است.
\end{definition}

\begin{theorem}
این یک قضیه است.
\end{theorem}
\end{frame}


\begin{frame}
    \begin{columns} 
     \column{.5\textwidth}
    ستون شماره ۱
     \column{.5\textwidth}
    ستون شماره ۲
    \end{columns}
    \end{frame}


\begin{frame}
\frametitle{Maths Blocks}
\begin{theorem}<1->[Pythagoras] 
\begin{align}
a^2 + b^2 = c^2
\end{align}
\end{theorem}
\begin{proof}<2->
$\omega +\phi = \epsilon $
\end{proof}
\begin{corollary}<3->
$ x + y = y + x  $
\end{corollary}
\end{frame}


\begin{frame}
کاربرد \lr{pause} در بیمر
\begin{itemize}
\pause
\item Point A
\pause
\item Point B
\begin{itemize}
\pause
\item part 1
\pause
\item part 2
\end{itemize}
\pause
\item Point C
\pause
\item Point D
\end{itemize}
\end{frame}

\begin{frame}
\frametitle{More Lists}
\begin{enumerate}[(I)]
\item<1-> Point A
\item<2-> Point B
\begin{itemize}
\item<3-> part 1
\item<4-> part 2
\end{itemize}
\item<5-> Point C
\item<6-> Point D
\end{enumerate}
\end{frame}


%\begin{frame}
%\frametitle{Overlays}
%\onslide<1->{First Line of Text}
%
%\onslide<2->{Second Line of Text}
%
%\onslide<3->{Third Line of Text}
%\end{frame}

\begin{frame}
\frametitle{Overlays}
\only<1>{First Line of Text}

\only<2>{Second Line of Text}

\only<3>{Third Line of Text}
\end{frame}






\setbeamercovered{invisible}
\begin{frame}
\frametitle{Tables}
\begin{table}
\begin{tabular}{l | c | c | c | c }
Competitor Name & Swim & Cycle & Run & Total \\
\hline \hline
John T & 13:04 & 24:15 & 18:34 & 55:53 \onslide<2-> \\ 
Norman P & 8:00 & 22:45 & 23:02 & 53:47 \onslide<3->\\
Alex K & 14:00 & 28:00 & n/a & n/a \onslide<4->\\
Sarah H & 9:22 & 21:10 & 24:03 & 54:35 
\end{tabular}
\caption{Triathlon results}
\end{table}
\end{frame}

\begin{frame}{عنوان اسلاید این صفحه  که در اینجا قرار می‌گیرد}
\begin{example}
این یک مثال است.
\end{example}
\end{frame}

\begin{frame}
\begin{Lemma}
متن متن متن متن متن متن متن متن متن متن متن متن متن متن متن متن متن متن متن متن متن متن متن متن متن متن متن متن متن متن متن متن متن 
\end{Lemma}
\end{frame}

\begin{frame}
\lr{http://qa.parsilatex.com/14100/}

\lr{http://qa.parsilatex.com/14148}
\end{frame}
